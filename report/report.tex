\documentclass{scrartcl}
\usepackage{rwukoma}
\usepackage[pdfusetitle]{hyperref}
\usepackage{caption}
\usepackage[backend=bibtex, style=numeric]{biblatex}
\addbibresource{bibliography.bib}

\title{LiDaR data recorded by a Velodyne VLP-16}
\author{AMOUSSOU Zinsou Kenneth}
\date{\today}

\begin{document}
	\maketitle
	\tableofcontents

	\clearpage
  
  \section{Introduction}

  Astyx Hires dataset \cite{astyx} is composed of 546 entries of measurements
  recorded with three sensors (camera, lidar and radar sensors). Each entry of
  the dataset contains samples of recording from the sensors at different
  timestamp.

  On the "Ego" vehicle, each sensor is mounted at a different position
  defining the center of its coordinate system.
  In order to provide a common base of analysis and interpretation of the
  dataset, each entry is associated to its calibration information
  (mainly transformation matrices). Thus, the measurements can transformed into
  different coordinate systems for analysis.

  As the main goal of the dataset is to be used for object detection
  \cite{astyx}, it also contains labeled objects (cars in its current
  iteration) designated by the term "Ground truth". The ground truth of each
  entry of the dataset contains details such as: label class name, oclusion,
  dimensions, orientation, etc \cite{astyx-spec}.

  Our goal in this report is to investigate how lidar sensors work by focusing
  on the case of Velodyne VLP-16. In a first step, a simplified model of a car
  is defined and estimations of lidar points coverage on the model are
  performed in different scenarios. Then a validation of our model has been  
  performed by checking the gap between estimations and real-life data from the
  astyx dataset.

  \section{LiDaR coverage estimation}
    \subsection{Simplified model of a car}

    Let's model a car as a 3D box. As simple that model can be, the first level
    of difficulty is to define the dimensions of that box. No all cars have the
    same dimensions. Even from the same manufacturer, different car designs
    can be found.
    Table \ref{table:car-sizes} show an overview of the diversity of car
    dimensions based on the car's type.

    \begin{table}[!htbp]
      \centering
      \begin{tabular}{ | c | c | c | c |}
        \hline
        \textbf{Type} & \textbf{Length (cm)} & \textbf{Width (cm)} & 
        \textbf{Height (cm)}  \\
        \hline \hline
        City cars & 269.5 - 366.5 & 147.5 - 166.5 & 146 - 161 \\
        \hline
        Superminis & 382.1 - 408.4 & 166.5 - 178 & 141.4 - 157.8 \\
        \hline
        Hatchbacks & 442.5 - 472.6 & 170.3 - 187.1 & 141.8 - 157.5 \\
        \hline
        Large cars & 423.6 - 496.6 & 169.3 - 189.5 & 142.9 - 155 \\
        \hline
        MPVs & 406.8 - 513 & 169.5 - 192.8 & 153 - 186 \\
        \hline
        SUVs & 466.2 - 513 & 176 - 200.8 & 162.4 - 203.5 \\
        \hline
      \end{tabular}
      \caption{Average car sizes \cite{car-sizes}}
      \label{table:car-sizes}
    \end{table}

    By computing the arithmetic average of each dimension in table
    \ref{table:car-sizes} as defined by equation x, we can settle the
    dimensions of the simplified model of the car as:

    \begin{equation}
      \centering
      D = \frac{1}{N} \sum_{n=1}^{N}{\frac{d_n(min) + d_n(max)}{2}}
      \label{equation:average-size}
    \end{equation}

    With $D$ the dimension, $N$ the total number of entries in the table -
    Here $N = 6$. $d_n(min)$ and $d_n(max)$ are respectively the minimal and
    maximal value of each entry of the table.

    The model of 3D box have the following dimensions as defined by
    \ref{equation:3d-model-dimensions}.

    \begin{equation}
      \centering
      length = 100 cm, 
      width = 100 cm, 
      height = 100 cm
      \label{equation:3d-model-dimensions}
    \end{equation}

    \subsection{Model positionning}

    Let's assume that the 3D model of the car considered is referenced in
    the same coordinate frame as the lidar sensor; and let's consider the
    sensor as the origin of the coordinate system. With those two hypothesis,
    it's important to define the transformation matrix that would allow
    achieving rotation of the model and its translation (moving away from the
    sensor).

  \section{Reduced model of a vehicule}

  \printbibliography
\end{document}
